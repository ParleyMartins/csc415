\documentclass[12pt]{article}
\usepackage{fullpage}
\usepackage{epsf}
\usepackage{graphicx}
\usepackage{listings}
\usepackage{float}
\lstset{
   breaklines=true,
   basicstyle=\ttfamily}

\input{commands}

\begin{document}

\noindent
Parley Pacheco Martins 1484000\\
AUCSC 415 -- Algotithms, Automata and Complexity\\
Winter 2016\\
Department of Science\\
University of Alberta, Augustana Faculty

\vspace*{0.75\baselineskip}
\hrule
\vspace*{0.75\baselineskip}

\noindent
{\Large\bf Case Study: elearning tools for computing science}

%%%%%%%%%%%%
\section{Introduction}

With the advance of the internet, more and more people are looking for ways to learn new things. One of these is through e-learning tools. These tools provide a method to learn and be evaluated without a physical classroom. Teachers and students from all over the world can share their knowledge as if they were together.

One of the most popular tools for this purpose is Moodle. A free software that allows teachers to upload content and grade their students based on several methods, like quizzes, open questions and file submission. It is a powerful e-learning environment, but it doesn't have everything teachers on a computing science course would need.

Courses related to programming oftten have assignments that are code based. Ususally, the final code is not so important and teachers that use Moodle can't 

\subsection{Git}

\subsection{Moodle}

%%%%%%%%%%%%
\section{Results}

%%%%%%%%%%%%
\section{Performance Discussion}

%%%%%%%%%%%%
\section{Other Approaches}

%%%%%%%%%%%%
\section{Conclusion}

\end{document}
