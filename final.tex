\documentclass[12pt]{article}
\usepackage{fullpage}
\usepackage{epsf}
\usepackage{graphicx}
\usepackage{listings}
\usepackage{float}
\lstset{
   breaklines=true,
   basicstyle=\ttfamily}

\input{commands}

\begin{document}

\noindent
Parley Pacheco Martins 1484000\\
AUCSC 415 -- Algotithms, Automata and Complexity\\
Winter 2016\\
Department of Science\\
University of Alberta, Augustana Faculty

\vspace*{0.75\baselineskip}
\hrule
\vspace*{0.75\baselineskip}

\noindent
{\Large\bf Case Study: elearning tools for computing science}

%%%%%%%%%%%%
\section{Introduction}

With the advance of the internet, more and more people are looking for ways to learn new things. One of these is through e-learning tools. These tools provide a method to learn and be evaluated without a physical classroom. Teachers and students from all over the world can share their knowledge as if they were together.

One of the most popular tools for this purpose is Moodle. A free software that allows teachers to upload content and grade their students based on several methods, like quizzes, open questions and file submission. It is a powerful e-learning environment, but it doesn't have everything teachers on a computing science course would need.

Courses related to programming oftten have assignments that are code based. Learning a new programming language, algorithm or technique is something that isn't always measured by the final code. A lot of that lives in the process and in the way the student thinks about the given problem and Moodle doensn't allow a teacher to see the development of any code.


%%%%%%%%%%%%
\section{Important Previous Concepts}

To know a programming language makes it easier to understand this paper, but it's not required though. In this sections a few key concepts are explained.

\subsection{Git}

Git is a popular Distributed Version Control System (DVCS). It stores a local copy of your repository making it really fast and powerful. Almost all operations are offline and it's easier to create or merge branches. \ref{bib:git}


%%%%%%%%%%%%
\section{Case Study}

To compare both tools the following example assignment will be used: 

Implement the Algorithm X (\ref{bib:dancing_links}) in any chosen language and analize how this approach can be better than the regular backtracking.

\subsection{GitHub Classroom}

GitHub Classroom is a tool that allows users to create virtual environments, classrooms, and set assignments based on GitHub. The instructor can create group and individual assignments and follow up on the changes they made (thoughout all the process). It's an open source software and anyone can have their own server running it. For this example a local copy of Classroom was used.

To get it running for the first time, do the following steps:

\begin{enumerate}

\item Assuming you have an account on GitHub, create an organization (we will be using AUCSC415).

\item Create an repository with the base code that your students will use. This is not a required step, however you don't have any other space to explain what is required from your students. A good practice is to describe it in you README.md file.

\end{enumerate}

\subsection{Moodle}

Moodle is well known e-lerning tool. It allows you to create your own courses and manage them with a lot of options in the most varied content. Moodle is also open source and can be customized as required.


%%%%%%%%%%%%
\section{Other Approaches}

%%%%%%%%%%%%
\section{Conclusion}

\end{document}
