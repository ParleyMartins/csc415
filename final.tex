\documentclass[12pt]{article}
\usepackage{fullpage}
\usepackage{epsf}
\usepackage{graphicx}
\usepackage{listings}
\usepackage{float}
\usepackage{hyperref}

\usepackage[
backend=biber,
style=ieee,
sorting=none,
citestyle=ieee
]{biblatex}
\addbibresource{references.bib}

\lstset{
   breaklines=true,
   basicstyle=\ttfamily}

\begin{document}

\noindent
Parley Pacheco Martins 1484000\\
AUCSC 415 -- Algotithms, Automata and Complexity\\
Winter 2016\\
Department of Science\\
University of Alberta, Augustana Faculty

\vspace*{0.75\baselineskip}
\hrule
\vspace*{0.75\baselineskip}

\noindent
{\Large\bf Case Study: E-learning Tools for Computing Science}

%%%%%%%%%%%%
\section{Introduction}

With the advance of the Internet, more and more people are looking for ways to learn new things. One of these is through e-learning tools. These tools provide a method to learn and be evaluated without a physical classroom. Teachers and students from all over the world can share their knowledge as if they were together.

One of the most popular tools for this purpose is Moodle. A free software that allows teachers to upload content and grade their students based on several methods, like quizzes, open questions and file submission. It is a powerful e-learning environment, but it doesn't have everything teachers on a computing science course would need.

Courses related to programming often have assignments that are code based. Learning a new programming language, algorithm or technique is something that isn't always measured by the final code. A lot of that lives in the process and in the way the student thinks about the given problem and Moodle doesn't allow a teacher to see the development of any code.


%%%%%%%%%%%%
\section{Important Previous Concepts}

To know a programming language makes it easier to understand this paper, but it's not required though. In this sections a few key concepts are explained.

\subsection{Git}

Git is a popular Distributed Version Control System (DVCS). It stores a local copy of your repository making it really fast and powerful. Almost all operations are offline and it's easier to create or merge branches. \cite{git}


%%%%%%%%%%%%
\section{Case Study}

To compare both tools the following example assignment will be used: 

Implement the Algorithm X (\cite{dancinglinks}) in any chosen language and analyze how this approach can be better than the regular backtracking.

\subsection{GitHub Classroom}

GitHub Classroom is a tool that allows users to create virtual environments, classrooms, and set assignments based on GitHub. The instructor can create group and individual assignments and follow up on the changes they made (throughout all the process). It's an open source software and anyone can have their own server running it. The official GitHub Classroom build was used on this comparison.

To get it running for the first time, do the following steps:

\begin{enumerate}

\item Assuming you have an account on GitHub, create an organization (we will be using AUCSC415 from now on to relate to an specific organization. Yours can have any name you want).

\item Access \href{https://classroom.github.com}{Classroom} with your GitHub credentials. 

\item Click on `Create your first classroom'. After clicking here you will need to grant access to one or more of your organizations. If you didn't create an organization in the previous step, you need to create one now.

\item Select AUCSC415. Change the name if you want. This is NOT the organization name, so you can add spaces and give a more descriptive name to it.

\item Follow the instructions on the screen if you want to add other administrators to this classroom.

\end{enumerate}

Now you have a classroom to manage your assignments and students. After doing that, you must create assignments.

\begin{enumerate}

\item In your GitHub Organization, create a repository with the base code that your students will use. This is not a required step, however you don't have any other space to explain what is required from your students. A good practice is to give any needed instructions in your README file, even if you don't have an initial code for your students.

\item Go to your classroom in GitHub Classroom and click on `Create your first assignment'. You can choose an individual or a group assignment. Group assignments can have a maximum number of people participating on them. This example will use individual assignment.

\item Give it a title and the repository where your starter code is. You can create private repositories depending on your GitHub plan.

\item Give the shown link to your students. 

\end{enumerate}

Once your students accept the request to create their assignments, a new repository (assignment-name-UserName) will be created in your organization. On your classroom dashboard you can see who accepted the request and check each repository to see their development.


\subsection{Moodle}

Moodle is a well known e-learning tool. It allows you to create your own courses and manage them with a lot of options in the most varied content. It is also open source and can be customized as required. The UofA eClass (version 2.8) was used to for this example.

To replicate the following steps your must have access to Moodle and be able to edit at least one course. \url{Moodle.com} provides a demo section if you like to test them.

Create a new assignment in your course by doing the following:

\begin{enumerate}

\item Create a new topic or choose an existing one in your course and turn edit on.

\item Click on `Add an activity or resource` and choose Assignment

\item Add any desired configuration like deadline, the necessity of pressing a submit button, the grading system. 

\end{enumerate}

After these steps, if your assignment was created on a public topic, it will be available to your students and they can submit their work according to the configurations you have set. If the topic is not available, your students won't be able to see this assignment.


%%%%%%%%%%%%
\section{Results}

\subsection{Assignment Creation}

To create anything in Moodle is a painful process. There are a lot of options and similar commands, and if you never used it before, it will take a good time to figure out how things work. It couldn't find a way to create a new topic, so I used an existing one and it still took me a few minutes to find a way to edit it and change its name.

To get on the `New Assignment` page was easier, but once there, there are so many options that you end up getting lost on what is better or what you should do. The standard configuration, however, gives an assignment that allows students to upload one single file with only one attempt.

In the other hand, it was way easier to create an assignment on GitHub Classroom. Once you have your classroom set up, the only options are to create a group or an individual assignment. No topics, no other divisions, no further options.

\subsection{Starter code}

Giving starter code to students is sometimes totally necessary to an assignment. The easy way this is done in GitHub Classroom is a benefit for this system. Despite that, once a student has accepted a request, it's not possible to change the initial code anymore, so that's a negative point for Classroom. If you do change it students that have accepted before that will have outdated code (or instructions).

Moodle has the same problem here, just in a different point of view. When you create your assignment, you can upload as many files as you need. You can edit those files anytime, but if a student has already downloaded them, it's really unlikely that he or she will open the assignment files again. Also, editing in Moodle can be complicated if you are not used to the platform (you have to turn editing on). Another downside of Moodle is that the files are not versioned and you can't know the real difference between them unless you download and carefully read the new file.

\subsection{Follow up on progress}

Moodle doesn't allow any king of follow up or check progress of a given assignment. Either your students will have submitted the assignment or not.

This is probably one of the biggest advantages of Classroom over Moodle. Because the work is on GitHub, a teacher can see anything he or she wants to see. They can comment and give specific feedback on the student implementation.

\subsection{Submitting}

While Classroom had advantage on the previous topics, it loses here. There is no way to submit anything on Classroom yet (that may change in the near future with Google Summer of Code, but this doesn't concern this paper). The only way to have a ``submission'' is by locking a branch after certain day. Or maybe adding (manually) a label to a branch or commit. Or even by creating a tag. But none of this can be done automatically and they all require that either the student or the teacher take an action.

Moodle was made with this purpose, so it gives you not only one, but plenty of options to submit assignments. You can specify a deadline, the way the submission will be made (one or multiple files and attempts), if an on-line text is required and how many words a student can add to it.

\subsection{Grading}

Grading is another problem for Classroom. There is no possible way to grade a student, at least not a conventional way. A teacher can create a label for this case too, but this information is more sensitive and usually not public, so the way out would be manually grading and sending it to the student (via e-mail maybe).

One more time Moodle shows why it's one of the most used e-learning tools in the world. During the creation of the assignment, it's possible to say how you want to grade it. You can set up rubrics, grade type or different scales. They can all be customized and you can also have templates to use in the future.

%%%%%%%%%%%%
\section{Conclusion}

GitHub Classroom is still very new (not even one year old) and because of that, it doesn't have a lot to offer. The idea is good and needed, but they didn't have time to implement everything a teacher requires when using such a tool. As shown in the results it has a very simple and intuitive interface, but it's still very incomplete.

Moodle is a tool that has millions of users \cite{moodlestats} and hundreds of contributors all over the world.\cite{moodledev} They worked and keep working to make plugins and improve the core features since 2002, what has led to a very robust and powerful tool. However it's not the right tool to check the how people are doing stuff. It focus on the final product (as usually in school) and the process to get there can't be seen.

If the main purpose of the teacher is to evaluate how a student is doing something, focused on making the creation process clearer and clarifying doubts about the new technique that is being taught, he or she should use GitHub Classroom. If a more traditional approach is used and just the final code matters, a instructor should use Moodle. Grading and submitting assignments are very powerful Moodle modules and they meet most of the necessities a teacher has concerning those issues.

\printbibliography %Prints bibliography
\end{document}
